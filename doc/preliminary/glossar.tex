\phantomsection
\addcontentsline{toc}{chapter}{Glossar}
\chapter*[Glossar]{Glossar}

\begin{glossary}
    \gls{Python}{Python ist eine Programmiersprache, die sowohl objektorientierte als auch funktionale Programmierung unterstützt. Des Weiteren kann Python als Scriptsprache verwendet werden.}
    \gls{LiDAR}{Steht für \textit{light detection and ranging}. Hierbei werden Laserimpulse ausgesandt und das zurückgestreute Licht detektiert, um Fernmessungen durchzuführen.}
    \gls{Video-Frame}{Ein einzelnes Bild einer Bildsequenz (=Videosequenz).}
    \gls{Odometrie}{Odometrie beschreibt ein Schätzverfahren der Position und Orientierung eines Objektes. Hierbei werden Rückschlüsse aus der Radumdrehung gezogen.}
    \gls{Morphologische Bildverarbeitung}{Beschreibt ein Modell für digitale Bilder, welches auf Verbandstheorie und Topologie basiert. \cite{MM}}
    \gls{Universally Unique Identifier}{Eine als hexadezimal notierte, eindeutige 16-Byte Identifizierungsnummer.}
    \gls{Hex-Farbcode}{Hexadezimale Repräsentation der Rot-, Grün- und Blau-Kanalwerte einer Farbe.}
\end{glossary}
