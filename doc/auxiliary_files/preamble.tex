%*******	 Package-Import		*******%
%Sprache
\usepackage[main=ngerman]{babel} %Sprache
\usepackage[utf8]{inputenc} %direkte Eingabe von Umlauten
\usepackage[T1]{fontenc} %Schriftencoding
\usepackage{hyphenat} %Silbentrennung

%Schrift
\usepackage{palatino} 
\usepackage{microtype} %typographische Perfektion
\usepackage{lmodern} %modernes Schriftbild
\usepackage{pifont}

%Bilder/Graphiken/Listings/Gleichungen
\usepackage{graphicx} %einbinden von Bildern/Grafiken
\usepackage{epstopdf}
\usepackage[section]{placeins}
\usepackage[hang]{caption}
%\usepackage{subcaption}
%\usepackage{nicefrac}
%\usepackage{listings} %Codeblöcke
\usepackage{minted}
\usepackage{mathtools} %Gleichungen
\usepackage{amsmath} % mehrzeilige Gleichungen
\usepackage{longtable}
\usepackage{multirow}
\usepackage{makecell}
\usepackage{adjustbox}

%Verzeichnisse
\usepackage[printonlyused]{acronym}
\usepackage{datatool} %zum sortieren eigenes Glossar

%Zitieren/Quellenangaben/Verweise
\usepackage[autostyle=try, style=german]{csquotes}
\usepackage[style=ieee, language=ngerman, block=ragged]{biblatex}
%\usepackage{url} %URL Angabe
\usepackage[hidelinks]{hyperref} %Querverweise
\usepackage{hypcap} %richtige "Sprungaddresse" für Bilder

%Sonstiges
\usepackage{xspace}
\usepackage{nextpage}

\setcounter{tocdepth}{4}
\setcounter{secnumdepth}{4}


%*******	 Wortumdefinierungen		*******%
\renewcommand*{\chapterautorefname}{Kapitel}
\renewcommand*{\sectionautorefname}{Abschnitt}
\renewcommand*{\subsectionautorefname}{Unterabschnitt}
\renewcommand*{\figureautorefname}{Abbildung}
\renewcommand*{\equationautorefname}{Gleichung}
\renewcommand*{\tableautorefname}{Tabelle}
\newcommand{\ctextit}[1]{\mbox{\textit{#1}}}

%*******	 Glossar		*******%
\newcommand{\gls}[2]{%
  \DTLnewrow{glossar}% Create a new entry
  \DTLnewdbentry{glossar}{label}{#1}% Add entry and description
  \DTLnewdbentry{glossar}{description}{#2}
}

\renewenvironment{glossary}{%
  \DTLifdbexists{glossar}{\DTLcleardb{glossar}}{\DTLnewdb{glossar}}% Create new/discard old glossary
}{%
  \DTLsort{label}{glossar}% Sort glossary
  \begin{description}
    \DTLforeach*{glossar}{\theLabel=label, \theDesc=description}{%
      \item[\theLabel] \hfill \\ \theDesc}% Print each item
  \end{description}%
}

%*******	 Einstellungen		*******%
%Dokument
\emergencystretch = 2.5em  %lässt größere Abstände zwischen Wörtern zu
\frenchspacing 
\OnehalfSpacing
\maxsecnumdepth{subsection}
\setlength\parindent{0pt}
\setlength{\parskip}{2ex}

%Penalties
\linepenalty=10     %page break within a paragraph
\hyphenpenalty=500  %line break at an automatic hyphen
\binoppenalty=700   %line break at a binary operator
\relpenalty=500     %line break at a relation
\clubpenalty=10000    %page break after first line of paragraph
\widowpenalty=10000   %page break before last line of paragraph
\brokenpenalty=1000  %page break after a hyphenated line
\tolerance=10      %acceptable badness of lines after hyphenation
\sloppybottom

%Seitenlayout
\settrims{0pt}{0pt}
\setxlvchars[\normalfont]
\setlxvchars[\normalfont]
\setlrmarginsandblock{30mm}{25mm}{*}
\setulmarginsandblock{30mm}{30mm}{*}
\setheadfoot{15mm}{10mm}
\checkandfixthelayout

%Kapitel
\makechapterstyle{black}{
    \chapterstyle{southall}
    \renewcommand\chapterheadstart{\vspace*{0pt}}
    \renewcommand*{\printchapternum}{%
        \addtolength{\midchapskip}{-\beforechapskip}
        \begin{minipage}[t][\baselineskip][b]{\beforechapskip}
            {\vspace{0pt}\chapnumfont\thechapter}
        \end{minipage}}
    \renewcommand\printchaptertitle[1]{%
        \hfill\begin{minipage}[t]{\midchapskip}
            {\vspace{0pt}\chaptitlefont ##1\par}
        \end{minipage}}
    \renewcommand\chaptitlefont{\huge\rmfamily\raggedright\bfseries}
    \setlength{\afterchapskip}{2ex}
    \addtolength{\midchapskip}{\beforechapskip}
}

\chapterstyle{black}



%header, footer und Co.
\makepagestyle{black} 
\makeoddfoot{black}{}{}{\ifonlyfloats{}{\thepage}} 
\makeevenfoot{black}{\ifonlyfloats{}{\thepage}}{}{}
\makeheadrule{black}{\textwidth}{\ifonlyfloats{0pt}{\normalrulethickness}}
\makeevenhead{black}{\ifonlyfloats{}{\small\textsc{\leftmark}}}{}{} 
\makeoddhead{black}{}{}{\ifonlyfloats{}{\small\textsc{\rightmark}}}
%\makeoddfoot{black}{}{}{\thepage} 
%\makeevenfoot{black}{\thepage}{}{}
\makeheadrule{black}{\textwidth}{\normalrulethickness}
\makeevenhead{black}{\small\textsc{\leftmark}}{}{} 
\makeoddhead{black}{}{}{\small\textsc{\rightmark}}

\makepsmarks{black}{
  \nouppercaseheads
  \createmark{chapter}{both}{shownumber}{}{\space}
  \createmark{section}{right}{shownumber}{}{\space}
}

\makepagestyle{blackchapter}
\makeoddfoot{blackchapter}{}{}{\thepage} 
\makeevenfoot{blackchapter}{\thepage}{}{}

\pagestyle{black}

%*******        Commands        *******%
\newcommand{\logos}[2]{\newcommand{\theLogos}{
        \begin{center}
            \includegraphics[scale=#2]{#1}
        \end{center}
    }}

\newcommand{\modul}[1]{\newcommand{\theModul}{#1\xspace}}
\newcommand{\faculty}[1]{\newcommand{\theFaculty}{#1\xspace}}
\newcommand{\institut}[1]{\newcommand{\theInstitut}{#1\xspace}}
\newcommand{\studentID}[1]{\newcommand{\theStudentID}{#1\xspace}}
\newcommand{\course}[1]{\newcommand{\theCourse}{#1\xspace}}
\newcommand{\supervisor}[1]{\newcommand{\theSupervisor}{#1\xspace}}
\newcommand{\secSupervisor}[1]{\newcommand{\theSecSupervisor}{#1\xspace}}

\newcommand*{\quelle}[1]{\par\vspace{1mm}\raggedleft\footnotesize Quelle:~#1\par\vspace{-3mm}}

\newcommand{\blankpage}{ 
    \newpage
    \thispagestyle{empty}
    \cleartooddpage[\thispagestyle{empty}] 
}

\newcommand{\initAnhang}{
    \renewcommand{\thepage}{\Alph{chapter}\ \arabic{page}}
    \newpage
    \setcounter{page}{1}
}

\newcommand{\longpage}[1][1]{\enlargethispage{#1\baselineskip}} 
% To increase the amount of text that can fit onto one page
% Ex. \longpage => increase \textheight with 1 row of text
% Ex. \longpage[2] => increase \textheight with 2 rows of text

\newcommand{\shortpage}[1][1]{\enlargethispage{-#1\baselineskip}} 
% To decrease the amount of text that can fit onto one page
% Ex. \shortpage => decrease \textheight with 1 row of text
% Ex. \shortpage[2] => decrease \textheight with 2 rows of text

\newenvironment{code}{\captionsetup{type=figure}}{}