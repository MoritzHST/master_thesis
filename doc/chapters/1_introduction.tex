\chapter{Einleitung}
\section{Motivation}
Das Autonome Fahren ist ein Konzept, welches mehr und mehr an Bedeutung gewinnt. Da mit der zunehmenden
Digitalisierung auch eine höherer Grad der Automatisierung gefragt ist, um z.B. Logistik- oder Transport-Prozesse 
kostengünstiger und sicherer zu gestalten, ergeben sich hierbei auch Anforderungen an das maschinengesteuerte Fahren.
Zum aktuellen Zeitpunkt kommt eine Hybrid-Form in modernen Fahrzeugen bereits zum Einsatz, wodurch unterschiedliche 
Fahrmannöver, wie z.B. das Einparken oder Halten einer Spur, teilautomatisiert werden. Damit das Fahren dauerhaft oder
vollständig von einem Softwaresystem übernommen werden kann, muss es auch in nicht-alltäglichen Situationen nachvollziehbar
entscheiden.

Um zwischen den einzelnen Ebenen der Autonomie zu unterscheiden, werden verschiedene Autonomiestufen definiert.
\begin{itemize}
    \item Autonomiestufe 0: "Driver only", der Fahrer fährt selbst ohne Assistenzsysteme.
    \item Autonomiestufe 1: Der Fahrer wird bei der Bedienung des Fahrezeugs unterstützt, wie z.B. durch einen Tempomat.
    \item Autonomiestufe 2: Das Fahrzeug ist teilautomatisiert und bietet Assistenzsysteme für automatisiertes Einparken 
                            oder zum halten der Spur.
    \item Autonomiestufe 3: Einzelne Fahrmannöver, wie z.B. das Wechseln der Fahrspur, werden vom Fahrzeug automatisiert durchgeführt.
                            Falls Handlungsbedarf für den Fahrer besteht, wird dieser innerhalb einer Vorwarnzeit zur Übernahme
                            der Fahrzeugführung aufgefordert. Aktuell wird darauf hingearbeitet, Fahrzeuge dieser Autonomiestufe
                            für den öffentlichen Straßenverkehr zuzulassen.
    \item Autonomiestufe 4: Das Softwaresystem übernimmt dauerhaft die Steuerung des Fahrzeugs. Falls der Fahrer die Fahrzeugführung
                            übernehmen muss, wird dieser innerhalb einer Vorwarnzeit benachrichtigt.
    \item Autonomiestufe 5: Das Fahrzeug ist vollautomatisiert, ein Fahrer ist nicht länger erforderlich. \cite{BASt}
\end{itemize}

Insbesondere für die Autonomiestufe 3 gab es bereits mehrere Projekte, die solche Systeme in der Praxis getestet haben. Im Juli 2014
gab es hierzu ein Pionierprojekt, bei welchem der Mercedes-Benz Future Truck 2025 auf einem gesperrten Autobahnteilstück bei Magdeburg 
autonom gefahren ist. Das System hat hierbei beispielsweise das mittige Fahren innerhalb der rechten Fahrspur sowie das Beschleunigen
und Bremsen übernommen. Der Fahrer konnte sich somit anderen Aufgaben widmen, wie z.B. der Planung der nächsten Tour oder auch der
Frachtkontrolle über digitale Displays. Durch den Wegfall mehrerer Bedienelemente, hat der Fahrzeugführer in diesem Beispiel auch mehr
Platz im Innenraum. \cite{benz-ft2025}

Zum Erreichen der Autonomiestufe 4 und 5 stellen sich hierbei neben der rechtlichen Grundlage auch technische Herausforderungen.
Ein Beispiel hierfür ist die Erkennung von Wildtieren, die den Straßenverkehr behindern oder gefährden können. Der schwedische
Autohersteller Volvo kann mit seiner Software beispielsweise die heimische Fauna bestehend aus Tieren wie Elchen, Rehen oder Rentieren
zuverlässig erkennen, scheitert jedoch beispielsweise an Kängurus. \cite{volvo-detection-report} 
Dies zeigt, dass es unzählig viele Szenarien gibt, die es beim Autonomen Fahren zu betrachten gibt, um die Fahrsicherheit entsprechend
zu gewährleisten. Hier gilt es auch darauf zu achten, dass die Sicherheit nicht nur für den Fahrzeugführer, sondern auch für Fußgänger, Radfahrer
u.Ä. gesichert ist. Aus diesem Grund werden in dieser Thesis eher Randszenarien behandelt, die nicht im alltäglichen Straßenverkehr auftreten jedoch
kritisch für die Sicherheit von Fahrzeugführer und Umwelt sind.

Abgesehen von öffentlichen Straßen gibt es auch andere Bereiche, in denen das Autonome Fahren Einzug erhalten hat. Ein Beispiel sind hier automatisierte
Transportsysteme, wie sie z.B. in Waren- oder Krankenhäusern zu finden sind. Die Roboter sind hier zuständig, sich innerhalb des eines Gebäudes zu orientieren und
einzelne Güter zu transportieren. Wichtig ist hierbei insbesondere, dass sich die Roboter nicht gegenseitig behindern oder zur Gefahr für den Menschen werden.

Die Orientierung in Gebäuden ist hierbei eine besondere Herausforderung, da es für diese im Regelfall keine digitalen Karten gibt, anhand der sich die Roboter
orientieren können. Diese Karten werden somit teils manuell erstellt, wodurch sich Probleme in der Genauigkeit sowie Flexibilität offenbaren. Der Vermessungsprozess kann darüber
hinaus mehrere Monate andauern, wie es z.B. im Greifswalder Klinikum der Fall war, als der Einsatz von TRANSCAR-Robotern vorbereitet wurde. \footnote{\url{https://webmoritz.de/2013/05/16/ameisen-im-klinikum/}}


\section{Zielstellung}
Für das autonome Fahren existieren schon zuverlässige Lösungen, die im Alltag funktionieren. In der Thesis wird deshalb der Fokus auf Randszenarien gelegt, die im Kern die Erkennung
und das Verhalten eines Fahrzeugs an einer ungeregelten Kreuzung beschreiben. Die Zielstellung ist, basierend auf dem Quellcode des "TurtleBot 3 AutoRace"-Projekts \footnote{\url{https://github.com/ROBOTIS-GIT/TurtleBot 3_autorace}},
ungeregelte Kreuzungen zu identifizieren. Der Roboter soll hier feststellen, wo die Kreuzung beginnt und in welche Richtungen er fahren kann. Zusätzlich soll er erkennen,
aus welchen Richtungen er mit Gegenverkehr rechnen muss. Umgesetzt wird dies mit Bildverarbeitung.
Im zweiten Schritt soll der Roboter zusätzlich erkennen, ob andere Fahrzeuge in der Nähe sind mit denen eine Kollisionsgefahr besteht. Hierfür wird eine Technik genutzt, die an Connected
Cars angelehnt ist.
Im letzten Schritt sollen beide Komponenten zusammengeführt werden. Der Roboter soll sich in seiner Umgebung so verhalten, dass grundlegende Regeln des Straßenverkehrs, wie z.B. rechts vor
links, umgesetzt werden. Darüber hinaus soll der Roboter auch potenzielle Regelbrüche, wie das nehmen der Vorfahrt, frühzeitig erkennen und eine Kollision vermeiden.

Grundsätzlich gab es in diese Richtung bereits Forschungen, jedoch waren diese Versuche auf Fahrzeuge innerhalb einer Fahrbahn begrenzt, bei denen versucht wurde, Kollisionen durch
diverse Mannöver zu vermeiden \cite{griffith}. Ein anderer Ansatz zu Erkennung von Kreuzungen war auf Basis von Punktwolken, die durch einen LiDAR-Sensor erstellt wurden. Dieser Ansatz
setzt jedoch eine physische Fahrbahnbegrenzung, wie durch einen Bordstein, voraus \cite{lidar-recognition}.

