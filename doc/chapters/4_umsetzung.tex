\chapter{Umsetzung}
\section{Umstrukturierung}
Der erste Bestandteil der Umsetzung ist es, das bestehende AutoRace-Projekt geringfügig umzustrukturieren. Im Wesentlichen sind hier zwei Komponenten relevant: 
die Entfernung des irrelevanten Quellcodes sowie die Umstrukturierung des Lebenszyklus-Managements der einzelnen Nodes.

Im bestehenden Projekt gibt es viel Programmcode, der für die Bearbeitung der Thesis grundlos ausgeführt wird. Im Konkreten handelt es sich hierbei um die Detektion von Ampeln
und Verkehrsschildern. Solche Infrastuktur-Elemente gibt es in den Simulationswelten der Thesis nicht, wodurch unnötige Rechenleistung beansprucht wird. Zusätzlich besteht
die Gefahr, dass diese Nodes für ein unvorhersehbares Verhalten sorgen, indem durch falsche Detektionen Daten veröffentlicht werden die zu einer Steuerung des Roboters führen. 
Aus diesem Grund werden die entsprechenden Komponenten entfernt. Zusätzlich werden die Core-Nodes soweit angepasst, dass sämtliche Fallunterscheidung bezüglich des Modus
auf das Wesentliche reduziert werden - das Halten der Fahrbahn und das Steuern an der Kreuzung. 

Das Lebenszyklus-Management der Nodes sieht aktuell vor, dass diese dynamisch gestartet und bei Bedarf komplett beendet werden. Dies hat zufolge, dass der Roboter bei der Transition
zwischen den einzelnen Modi warten muss, bis die Nodes wieder gestartet sind. Das ist ein unschönes Verhalten. In der Theorie gibt es für ROS ein Lebenszyklus-Management.\footnote{\url{https://design.ros2.org/articles/node_lifecycle.html}} 
Das bestehende Projekt setzt auf die \textit{roslaunch}-API, um Nodes dynamisch zu starten und zu beenden. Für diese wurde die Funktionalität nicht umgesetzt oder dokumentiert. 
Aus diesem Grund wird hier für jede Node ein neues Topic eingeführt. Über diese Topics wird mit Boolean-Nachrichten kommuniziert. Für die Node \textit{control\_crossing} wird beispielsweise
das Topic \textit{/control\_crossing/active} implementiert. Wenn hier beispielsweise \textit{\textbf{TRUE}} empfangen wird, werden alle Subscriber der Node registiert. Wenn jedoch ein \textit{\textbf{FALSE}}
empfangen wird, werden alle Subscriber der Node abgemeldet.
Hierdurch kann der Roboter fast nahtlos den Modus wechseln. Ein Nachteil dieses Vorgehens ist, dass so alle Instanzen einer Node in einem Namespace simultan deaktiviert werden.
Da für diese Thesis jedoch jede Node innerhalb ihres Namespaces einzigartig ist, ist dieser Sachverhalt unerheblich.