\documentclass[a4paper,12pt,oneside]{report}
\usepackage[utf8]{inputenc}
\usepackage{graphicx}
\usepackage[utf8]{inputenc}
\usepackage[T1]{fontenc}
\usepackage{textcomp}
\usepackage{listings}
\usepackage{hyperref}
\usepackage{xcolor}
\usepackage[nonumberlist]{glossaries}
\definecolor{listinggray}{gray}{0.9}
\definecolor{lbcolor}{rgb}{0.9,0.9,0.9}
\renewcommand{\chaptername}{}
\newcommand{\classname}[1]{\mbox{\textit{\allowbreak #1}}}
\usepackage[ngerman]{babel}
\usepackage{microtype}
\usepackage{lmodern}
\usepackage{longtable}
\usepackage{fancyhdr}
%\usepackage[round]{natbib}
\bibliographystyle{abbrvdin}
\usepackage{wrapfig}
\usepackage{listings}
\setcounter{tocdepth}{3}
\setcounter{secnumdepth}{3}
\lstset{
basicstyle=\footnotesize\ttfamily, % Default font
  % numbers=left,              % Location of line numbers
  numberstyle=\tiny,          % Style of line numbers
  % stepnumber=2,              % Margin between line numbers
  numbersep=5pt,              % Margin between line numbers and text
  tabsize=2,                  % Size of tabs
  extendedchars=true,
  breaklines=true,            % Lines will be wrapped
  keywordstyle=\color{red},
  frame=bt,
  showstringspaces=false,
  % keywordstyle=[1]\textbf,
  % keywordstyle=[2]\textbf,
  % keywordstyle=[3]\textbf,
  % keywordstyle=[4]\textbf,   \sqrt{\sqrt{}}
  stringstyle=\color{blue}\ttfamily, % Color of strings
  showspaces=false,
  showtabs=false,
  xleftmargin=17pt,
  framextopmargin=10pt,
  framexleftmargin=20pt,
  framexrightmargin=20pt,
  framexbottommargin=10pt,
  % backgroundcolor=\color{lightgray},
  showstringspaces=false
}

\makeatletter
\def\@makechapterhead#1{%
  \vspace*{50\p@}%
  {\parindent \z@ \raggedright \normalfont
    \interlinepenalty\@M
    \Huge\bfseries  \thechapter.\quad #1\par\nobreak
    \vskip 40\p@
  }}
\makeatother

\lstdefinelanguage{json}{
    basicstyle=\tiny\ttfamily,
    numbers=left,
    numberstyle=\scriptsize,
    stepnumber=1,
    numbersep=8pt,
    showstringspaces=false,
    breaklines=true,
    frame=lines,
    backgroundcolor=\color{background},
    literate=
     *{0}{{{\color{numb}0}}}{1}
      {1}{{{\color{numb}1}}}{1}
      {2}{{{\color{numb}2}}}{1}
      {3}{{{\color{numb}3}}}{1}
      {4}{{{\color{numb}4}}}{1}
      {5}{{{\color{numb}5}}}{1}
      {6}{{{\color{numb}6}}}{1}
      {7}{{{\color{numb}7}}}{1}
      {8}{{{\color{numb}8}}}{1}
      {9}{{{\color{numb}9}}}{1}
      {:}{{{\color{punct}{:}}}}{1}
      {,}{{{\color{punct}{,}}}}{1}
      {\{}{{{\color{delim}{\{}}}}{1}
      {\}}{{{\color{delim}{\}}}}}{1}
      {[}{{{\color{delim}{[}}}}{1}
      {]}{{{\color{delim}{]}}}}{1},
}

\makeglossaries
\newglossaryentry{HTTP}
{
    name=HTTP,
    description={Hypertext Transfer Protocol, wird zur zustandslosen Übertragung von Daten
    genutzt, wie zum Beispiel zur Übertragung von Webseiten}
}

\newglossaryentry{EDIFACT}
{
    name=EDIFACT,
    description={Electronic Data Interchange for Administration, Commerce and Transport,
    eine Norm, die als Grundlage zum länder- und branchenübergreifenden Datenaustausch dient}
}

\newglossaryentry{XPath}
{
    name=XPath,
    description={XML Path Language, wird genutzt, um Elemente in XML-Strukturen zu identifizieren}
}

\newglossaryentry{DNSSpoofing}
{
    name=DNS-Spoofing,
    description={Angriffsmethode, bei der einem Client eine inkorrekt aufgelöste IP-Adresse des eigentlich
    angefragten Ziels übermittelt wird}
}

\newglossaryentry{CRUD}
{
    name=CRUD,
    description={Create, Read, Update, Delete}
}

\newglossaryentry{UUID}
{
    name=UUID,
    description={Universally Unique Identifier, Standard zur eindeutigen Identifizierung
    von Ressourcen}
}


\newglossaryentry{JSON}
{
    name=JSON,
    description={JavaScript Object Notation, Datenformat zum Datenaustausch}
}

\newglossaryentry{CICD}
{
    name=CI/CD,
    description={Continuous Integration, Continuous Delivery, beschreibt die kontinuierliche
    Integration von Entwicklungsartefakte in ein Projekt und deren Implementierung in eine Produktivumgebung}
}


\begin{document}

% Deckblatt >>
\begin{titlepage}
    \begin{center}
        \vspace*{1cm}

        \Huge
        \textbf{Masterthesis}

        \vspace{0.5cm}
        \LARGE
        Simulation und Erprobung verschiedener Szenarien des Autonomen Fahrens am TurtleBot3 

        \vspace{0.75cm}

        \textbf{Moritz Wilke}

        \vspace{0.8cm}

        \includegraphics[width=0.4\textwidth]{images/host.png}

        \Large
        Hochschule Stralsund\\
        Fakultät für Elektrotechnik und Informatik\\
        \vspace{0.5cm}
        Informatik (INFM)\\
    \end{center}
    \vfill
    \begin{minipage}{0.45\linewidth}
    \begin{flushleft}
       \small
       Erstgutachter:\\
       Prof. Dr. rer. nat. Christian Bunse\\
       \vspace{0.5cm}
       Zweitgutachter:\\
       Dr.-Ing. Jöran Pieper
    \end{flushleft}
    \end{minipage}
    \hfill
    \begin{minipage}{0.45\linewidth}
    \begin{flushright}
      \small
      Matrikel-Nr.:\\
      15214\\
      \vspace{0.5cm}
      Abgabedatum:\\
      06.10.2021
    \end{flushright}
    \end{minipage}
\end{titlepage}
% Deckblatt <<

% Inhaltsverzeichnis >>
\renewcommand{\contentsname}{Inhaltsverzeichnis}
\pagenumbering{gobble}
\tableofcontents
\clearpage
% Inhaltsverzeichnis <<

\pagenumbering{arabic}
\pagestyle{fancy}
\fancyhf{}
\fancyhead[LO]{Hochschule Stralsund - Moritz Wilke}
\fancyhead[CF]{\thepage}
\fancyhead[CH]{}
% Einzelne Kapitel
\chapter{Einleitung}
\fancyhead[RO]{\thechapter. Einleitung}
\section{Motivation}
Das Autonome Fahren ist ein Konzept, welches mehr und mehr an Bedeutung gewinnt. Da mit der zunehmenden
Digitalisierung auch eine höherer Grad der Automatisierung gefragt ist, um z.B. Logistik- oder Transport-Prozesse 
kostengünstiger und sicherer zu gestalten, ergeben sich hierbei auch Anforderungen an das maschinengesteuerte Fahren.
Zum aktuellen Zeitpunkt kommt eine Hybrid-Form in modernen Fahrzeugen bereits zum Einsatz, wodurch unterschiedliche 
Fahrmannöver, wie z.B. das Einparken oder Halten einer Spur, teilautomatisiert werden. Damit das Fahren dauerhaft oder
vollständig von einem Softwaresystem übernommen werden kann, muss es auch in nicht-alltäglichen Situationen nachvollziehbar
entscheiden.

Um zwischen den einzelnen Ebenen der Autonomie zu unterscheiden, werden verschiedene Autonomiestufen definiert.
\begin{itemize}
    \item Autonomiestufe 0: "Driver only", der Fahrer fährt selbst ohne Assistenzsysteme.
    \item Autonomiestufe 1: Der Fahrer wird bei der Bedienung des Fahrezeugs unterstützt, wie z.B. durch einen Tempomat.
    \item Autonomiestufe 2: Das Fahrzeug ist teilautomatisiert und bietet Assistenzsysteme für automatisiertes Einparken 
                            oder zum halten der Spur.
    \item Autonomiestufe 3: Einzelne Fahrmannöver, wie z.B. das Wechseln der Fahrspur, werden vom Fahrzeug automatisiert durchgeführt.
                            Falls Handlungsbedarf für den Fahrer besteht, wird dieser innerhalb einer Vorwarnzeit zur Übernahme
                            der Fahrzeugführung aufgefordert. Aktuell wird darauf hingearbeitet, Fahrzeuge dieser Autonomiestufe
                            für den öffentlichen Straßenverkehr zuzulassen.
    \item Autonomiestufe 4: Das Softwaresystem übernimmt dauerhaft die Steuerung des Fahrzeugs. Falls der Fahrer die Fahrzeugführung
                            übernehmen muss, wird dieser innerhalb einer Vorwarnzeit benachrichtigt.
    \item Autonomiestufe 5: Das Fahrzeug ist vollautomatisiert, ein Fahrer ist nicht länger erforderlich. \cite{BASt}
\end{itemize}

Insbesondere für die Autonomiestufe 3 gab es bereits mehrere Projekte, die solche Systeme in der Praxis getestet haben. Im Juli 2014
gab es hierzu ein Pionierprojekt, bei welchem der Mercedes-Benz Future Truck 2025 auf einem gesperrten Autobahnteilstück bei Magdeburg 
autonom gefahren ist. Das System hat hierbei beispielsweise das mittige Fahren innerhalb der rechten Fahrspur sowie das Beschleunigen
und Bremsen übernommen. Der Fahrer konnte sich somit anderen Aufgaben widmen, wie z.B. der Planung der nächsten Tour oder auch der
Frachtkontrolle über digitale Displays. Durch den Wegfall mehrerer Bedienelemente, hat der Fahrzeugführer in diesem Beispiel auch mehr
Platz im Innenraum. \cite{benz-ft2025}

Zum Erreichen der Autonomiestufe 4 und 5 stellen sich hierbei neben der rechtlichen Grundlage auch technische Herausforderungen.
Ein Beispiel hierfür ist die Erkennung von Wildtieren, die den Straßenverkehr behindern oder gefährden können. Der schwedische
Autohersteller Volvo kann mit seiner Software beispielsweise die heimische Fauna bestehend aus Tieren wie Elchen, Rehen oder Rentieren
zuverlässig erkennen, scheitert jedoch beispielsweise an Kängurus. \cite{volvo-detection-report} 
Dies zeigt, dass es unzählig viele Szenarien gibt, die es beim Autonomen Fahren zu betrachten gibt, um die Fahrsicherheit entsprechend
zu gewährleisten.


\section{Zielstellung}
Welche Szenarien gibt es zu beachten und wie sollen diese bearbeitet werden?


\chapter{Grundlagen}
\fancyhead[RO]{\thechapter. Grundlagen}
\section{Szenarien}
\subsection{ggf. schon potenzielle Anwendungsfälle aufführen}

\section{Softwaredesign}
\subsection{--Welche Softwaredesignprinzipien werden genutzt?--}
\subsection{Event-Driven Systems}
\subsection{Bildverarbeitung}
\subsection{Objekterkennung}
\subsection{Connected Cars}
\subsubsection{Vehicle to Vehicle (V2V)}
\subsubsection{Vehicle to Pedestrian (V2P)}
\subsubsection{Vehicle to Infrastructure (V2I)}
\subsubsection{Infrastructure to Vehicle(I2V)}

\section{Tools und Frameworks}
\subsection{ROS}
\subsection{Gazebo}
\subsection{TurtleBot3}

\chapter{Konzeption}
\fancyhead[RO]{\thechapter. Konzeption}
\section{Konzeption Szenario 1}
\section{usw...}

\chapter{Simulation}
\fancyhead[RO]{\thechapter. Simulation}
\section{Simulation Szenario 1 (wie geplant, wie Voraussetzung geschaffen, Ziel, Ergebnis usw.)}
\section{usw...}

\chapter{Erprobung}
\fancyhead[RO]{\thechapter. Erprobung}
\section{Erprobung Szenario 1 (wie geplant, wie Voraussetzung geschaffen, Ziel, Ergebnis usw.)}
\section{usw...}

\chapter{Ergebnisse}
\fancyhead[RO]{\thechapter. Ergebnisse}
\input{chapters/6_ergebnisse}

\chapter{Fazit und Ausblick}

\fancyhead[LO]{Hochschule Stralsund - Moritz Wilke}
\fancyhead[CF]{\thepage}
\fancyhead[CH]{}
\fancyhead[RO]{}

\glsaddall
\printglossaries

\bibliography{chapters/literature}

\newpage
\section*{Eidesstattliche Erklärung}
\pagenumbering{gobble}
Ich versichere, die von mir vorgelegte Arbeit selbständig verfasst zu haben. Alle Stellen, die wörtlich oder sinngemäß aus veröffentlichten oder nicht veröffentlichten Arbeiten anderer entnommen sind, habe ich als entnommen kenntlich gemacht. Sämtliche Quellen und Hilfsmittel sind angegeben. Die Arbeit hat mit gleichem bzw. in wesentlichen Teilen gleichem Inhalt noch keiner Prüfungsbehörde vorgelegen.

\vspace{1.5cm}
\begin{flushleft}
\begin{minipage}{0.45\linewidth}
  Stralsund, den ~\hrulefill
\end{minipage}
\hfill
\vspace{0.75cm}
\begin{minipage}{0.45\linewidth}
  \vphantom{Stralsund, den}\hrulefill \\
  Moritz Wilke
\end{minipage}
\end{flushleft}


\end{document}
